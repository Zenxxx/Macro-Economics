\documentclass[dvipdfmx,a4paper, 12pt]{article}
\usepackage{wrapfig, blindtext}
\usepackage{amsmath,amssymb,graphicx} %load extra symbols and environments
\usepackage[margin=1in]{geometry} %set margins
\usepackage{enumerate}
\usepackage{mathtools}
\usepackage{autobreak}
\usepackage{graphicx}
\usepackage{color}
\usepackage{url}
\usepackage{tcolorbox}
\usepackage{bm}
\usepackage{breqn}
\usepackage{here}
\usepackage{ascmac}
\usepackage{empheq}
\usepackage{setspace}
\usepackage{here}
\usepackage{mathrsfs}
\usepackage{mathrsfs}
\usepackage{bbm}
\usepackage{eurosym}

\hyphenpenalty=10000\relax
\exhyphenpenalty=10000\relax
\sloppy

\newcommand{\R}{\mathbb{R}}
\newcommand{\Z}{\mathbb{Z}}
\newcommand{\N}{\mathbb{N}}
\newcommand{\Q}{\mathbb{Q}}
\newcommand{\C}{\mathbb{C}}
\newcommand{\T}{\mathbb{T}}

\allowdisplaybreaks[4]
\tcbuselibrary{breakable, skins, theorems}

\newtcbtheorem{definition1}{Definition}{enhanced,
attach boxed title to top left = {xshift=5mm,yshift=-3mm},
boxed title style = {colframe = blue!35!black, colback = white},
coltitle = black,
colback = white,
colframe = blue!35!black,
fonttitle = \bfseries,
breakable = true,
top = 4mm
}{korehatheorem1}

\newtcbtheorem{theorem2}{Theorem}{enhanced,
attach boxed title to top left = {xshift=5mm,yshift=-3mm},
boxed title style = {colframe = green!35!black, colback = white},
coltitle = black,
colback = white,
colframe = green!35!black,
fonttitle = \bfseries,
breakable = true,
top = 4mm
}{korehatheorem2}

\newtcbtheorem{proof3}{Proof}{enhanced,
attach boxed title to top left = {xshift=5mm,yshift=-3mm},
boxed title style = {colframe = black!35!black, colback = white},
coltitle = black,
colback = white,
colframe = black!35!black,
fonttitle = \bfseries,
breakable = true,
top = 4mm
}{korehatheorem3}

\begin{document}

\setlength{\abovedisplayskip}{1pt}
\setlength{\belowdisplayskip}{1pt}

\onehalfspacing

\title{マクロ経済学}
\author{}
\date{}
\maketitle

\newpage

\tableofcontents

\newpage

\begin{Introduction}
ここでは大学一年生レベルのマクロ経済学(GDPなど基本的なことからIS-LMモデルまで)についていろいろ書き連ねていくよ。事実だけをポンポン投げてくスタイルじゃなくて計算したりグラフを積極的に使っていくよ。数学的な記述が多いと思うけど数式とかで理解したほうが意味は分かりづらいかもしれないけど問題解く時に役に立つからたくさん使っていくよ。最初の方は文字がいっぱい書いてあるけど分析を始めていくと数学とグラフをたくさん使うようになるよ。

そもそもあんまり日本語を書かないし、経済は決して得意な方じゃないから誤植とか間違いがあったら教えてね。

\newpage

\section{マクロ経済で重要な3変数}
\subsection{国内総生産(GDP)}
\subsubsection{GDPの定義と計算方法}
GDPはその名の通り1年間に自国で生み出された価値の総額だよ。これはつまり国の豊かさを表してるってことになるね。GDPの測定方法としては下の三つがあるけどどれも最終的に値が一致するように計算されるから方法で求めることができるよ。この三つの方法で別々に計算しても値が一致することを三面等価の原則って呼ぶよ。
\begin{enumerate}
  \item 生産法は市場に出回ってなおかつ直接消費される財の合計だよ。計算式だと
\begin{equation*}
    GDP=\sum_i p_iq_i
\end{equation*}
ここで$p_i$は財$i$の価格、$q_i$は財$i$の数量を表すよ。市場価格が適用されない財について:

\quad $\cdot$ 政府のサービスなどについては補完が行われるよ

\quad $\cdot$ 家事などは補完が行われないよ

この補完が行われるかどうかによって大きくGDPが変わってしまうことがあるよ。例えばA国では子供を保育園に預けて母親は仕事をする事が一般的でB国では子供を母親が一日中面倒を見るっていうことになると子供の面倒を見ることはGDPに価値として補完されないからA国のGDPだけが上がることになるね。

生産法では直接消費される財だけが計算式に含まれるよ。例えばパン屋さんが市場に出回っている小麦粉を買ってそれを使ってパンを生産したとするとこの小麦粉の値段はGDPの値には含まれないよ。

生産法を少し改良したものとして付加価値法があるよ。これはさっきのパン屋さんの例で言うと生産法だと最終的に生産されたパンの値段をGDPにするけど付加価値法では小麦粉の値段とパン屋さんの仕事にやって生み出された価値を足し合わせるよ。

  \item 所得法はその名の通り所得からGDPを計算する方法だよ。具体的には

GDP=従業員の所得+資本所得+混合所得+税で表せるよ。

ここで資本所得は労働をしないで得られる所得のことだよ。例えば土地を貸しているというのはお金になるけど実質的な業務は行っていないとか。普通の所得と資本所得は判断するのが難しい場合があるからそれを混合所得として数えるよ。

従業員の所得と資本所得はデータを集めて正しく計算するのが難しいから理論的には正しいけどあんまり使われないよ

  \item
支出法は一番メジャーなGDPの計算方法だよ。具体的には
\begin{equation*}
  GDP=C+I+G+NX
\end{equation*}
で表されるよ。$C$は家計消費、$I$は投資、$G$は政府支出、$NX$は輸出から輸入の額を引いたものだよ。例えば自動車会社があって自動車を100台生産してもそれがその年のうちに全部売れるとは限らなくて売れ残ることもあるよね。そういう時にGDPにこの車の価格を入れないと三面等価の原則が成り立たなくなっちゃうから企業のオーナーが全部買ったことにして消費に含めるよ。
\end{enumerate}
\subsubsection{GDPの比較}
実はGDPには二種類存在して一つを名目GDP、もう一つを実質GDPっていうよ。生産法のGDPの計算方法を思い出してほしいんだけど財の価格と数量でGDPを求めたよね。異なる年度でのGDPを比較するときには財の価格が上がったのか、それとも経済が成長してもっと多くの数量を生産できるようになったのかが気になるよね。名目GDPは以下の数式で表されるよ。
\begin{equation*}
  P_tY_t=\sum_i p_{i,t}q_{i,t}
\end{equation*}
ここで$P_tY_t$は時間$t$での名目GDP、$p_{i,t}$は時間$t$での財$i$の価格、$q_{i,t}$は時間$t$での財$i$の数量を表しているよ。例えばGDPの成長率を知りたいときとかに名目GDPを使ってしまうと生産量が上がったのか価格が上がったのかわからないから実質GDPってものを導入するよ。

実質GDPは二通りの計算方法があるよ。一つは基準年を一つ決めて財の価格をその年の価格で固定するというものだよ。つまり、
\begin{equation*}
  Y_t=\sum_i p_{i,0}q_{i,t}
\end{equation*}
ここで$Y_t$は時間$t$での実質GDP、$p_{i,0}$は基準年での財$i$の価格、$q_{i,t}$は時間$t$での財$i$の数量を表しているよ。これだと計算は簡単だけどGDPの成長率を求めるときに基準年をどう決めるかによって変わってしまうね。

もう一つの計算方法は日本語だと連鎖方式って言われるよ(英語だとchain-weighted GDP)。これの計算方法は少し複雑だけどGDP成長率は基準年の選び方に依らないよ。具体的には以下で計算するよ。
\begin{enumerate}
  \item 初めに時刻$t$と$t+1$での連鎖的成長率を二通り計算するよ。つまり$t$を基準年としたときの成長率を、$Y_t^t=p_{i,t}q_{i,t}$、$Y_{t+1}^t=p_{i,t}q_{i,t+1}$として$\displaystyle g^t=\frac{Y_{t+1}^t-Y_t^t}{Y_t^t}$とするよ。次に$t+1$を基準年として同じように計算すると$Y_t^{t+1}=p_{i,t+1}q_{i,t}$、$Y_{t+1}^{t+1}=p_{i,t+1}q_{i,t+1}$として$\displaystyle g^{t+1}=\frac{Y_{t+1}^{t+1}-Y_t^{t+1}}{Y_t^{t+1}}$
  となるね
  \item 次に平均連鎖的成長率を$g^{ch}$として以下で計算するよ
  \begin{equation*}
    1+g^{ch}=\sqrt{\left(1+g^1\right)\left(1+g^2\right)}
  \end{equation*}
  この式は
  \begin{equation*}
    g^{ch}\approx \frac{1}{2}\left(g^1+g^2\right)
  \end{equation*}
  という式で近似することができるよ。
  \item 最後にこの平均連鎖的成長率を用いて各時刻のGDPを以下の式で求めるよ。ここで基準年は$t$とするよ。
  \begin{displaymath}
    \begin{array}{l}
      Y_t^{ch,t}=p_{i,t}q_{i,t}\\
      Y_{t+1}^{ch,t} \approx Y_t^{ch,t}(1+g^{ch})
    \end{array}
  \end{displaymath}
\end{enumerate}

\subsubsection{国でのGDPの比較}
二国間のGDPの比較は架空通貨を導入することでできるようになるよ。
今、USAのGDPを$GDP_{USA}=\$10000$、日本のGDPを$GDP_{JP}=\text{Y\llap{=}}500000$とするよ。
\begin{enumerate}
  \item まず架空通貨、インターナショナルドル(I\$)を導入するよ。
  \item 次にUSAの財とサービスが少しずつ入っている1\$の「塊」を用意するよ。(つまりこの塊を10000個買えばUSAの国内総生産と一致する)。この「塊」はI\$1で買えるものとするよ。
  \item すでに$GDP_{USA}=\$10000=\text{I\$}10000$ということがわかるね。
  \item さっき作った「塊」を日本で買おうとしたらいくらになるかを算出するよ。(これは外生的なものだから計算方法とかはないよ)
  \item 今、この「塊」がY\llap{=}130すると仮定するとI\$1=Y\llap{=}130だから$GDP_{JP}=\text{Y\llap{=}}500000\approx \text{I\$}=3846$ってことがわかってこれで比較ができるようになったね。
\end{enumerate}

\subsection{インフレ率}
\subsection{失業率}
\end{document}
